\documentclass[11pt]{article}


%% Header stuff that you shouldn't need to change.
\usepackage{amsfonts}
\usepackage{amsmath}
\usepackage{amsthm}
\usepackage{amssymb}
\usepackage{fullpage}

\theoremstyle{plain} % Heading in bold, text in italics.
\newtheorem{theorem}{Theorem}[section]
\newtheorem{corollary}[theorem]{Corollary}
\newtheorem{lemma}[theorem]{Lemma}
\newtheorem{claim}[theorem]{Claim}
\newtheorem{proposition}[theorem]{Proposition}
\newtheorem{problemstatement}{Problem} % \problem is taken :)
\newtheorem{question}{Question}
\newtheorem*{lemma*}{Lemma}
\newtheorem*{claim*}{Claim}
\newtheorem*{theorem*}{Theorem}
\newtheorem*{proposition*}{Proposition}
\newtheorem*{problemstatement*}{Problem}
\newtheorem*{corollary*}{Corollary}
\theoremstyle{definition} % Bold header, roman text
\newtheorem{definition}[theorem]{Definition}
\newtheorem{conjecture}[theorem]{Conjecture}
\newtheorem{example}[theorem]{Example}
\newtheorem{notation}[theorem]{Notation}
\newtheorem{exercise}[theorem]{Exercise}
\newtheorem*{notation*}{Notation}
\newtheorem*{definition*}{Definition}
\newtheorem*{conjecture*}{Conjecture}
\newtheorem*{example*}{Example}
\theoremstyle{remark} % Italic header, roman text
\newtheorem{remark}[theorem]{Remark}
\newtheorem*{remark*}{Remark}
\newtheorem{case}{Case}
\newtheorem{answer}{Answer}
\newtheorem*{solution}{Solution}
\newtheorem*{hint}{Hint}

%% End of header stuff

\begin{document}
\title{MATH 431 HW xxx Initial Version}
\author{My Name}
\maketitle

% Lines starting with a % sign are comments and will not appear in the pdf

\section{Problem 3.4.5}

\begin{proposition*}
  Every function $f$ is blah blah blah.
  % Dollar signs enter math mode.  Every mathematical symbol and
  % variable should be in math mode.
\end{proposition*}

\begin{proof}
  If $f$ is blah and $x$ is blah blah then $3 \sin x + 4 f - 3$ is
  blah blah.  Also $\frac{2x+1}{4x+9}$ is blah blah blah.
  % Use \frac for fractions, much better than $(2x+1)/(4x+9)$.
  Here is a longer ``display'' equation that goes on its own line.
  \begin{equation*}
    \sqrt{\frac{x^2+9}{y-3}+7} +
    \lim_{n \to \infty} f_n(x) = \int_0^3 g(x)\,dx
    % \, inserts a small space between g(x) and dx.
  \end{equation*}
  Multi line equations can be done as follows:
  \begin{align*}
    |f_n(x) - f(x)| &= \left| \frac{2nx}{n+1} - 2x \right| \\
    &= \left| -\frac{2x}{n+1} \right| \\
    &< \epsilon
    % The & signs indicate what should be aligned.  \\ starts a new line.
  \end{align*}
  Now the proof is complete.
\end{proof}

% if you leave off the * you get a number you can refer back to.
\begin{proposition}\label{coolprop}
  A proposition with a number.
\end{proposition}

\begin{proof}
  Blah blah blah.
\end{proof}

Now let's refer back to Proposition \ref{coolprop}.

Here is an equation with a number that you can refer back to.
\begin{equation}\label{fancy}
  e^{i \pi} + 1 = 0
\end{equation}
Equation \eqref{fancy} is super fancy.

You can get ``blackboard bold'' with $\mathbb{R}, \mathbb{Q},
\mathbb{Z}$, etc.

\section{Another problem}

Et cetera.

\end{document}
