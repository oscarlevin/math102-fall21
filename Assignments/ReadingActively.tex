\documentclass[11pt]{exam}

\usepackage{amsmath, amssymb, multicol}
\usepackage{graphicx}
\usepackage{xcolor}
\usepackage{textcomp}
\usepackage{comment}
\usepackage{etoolbox} %for toggles
\usepackage[draft]{todonotes}
\usepackage{tikz}
\usepackage{url}


% % % % % % % % % Create instructor edition environment % % % % % % % % % % % % %
\specialcomment{instnote}{\begingroup\vskip 1em \color{blue} \noindent{\bfseries Notes: }}{\vskip 1em\endgroup}
\newcommand{\hideinstnotes}{\excludecomment{instnote}} %by including \hideinstnotes, all instructor notes will be hiden.

\newenvironment{question}{\question}{}

\def\d{\displaystyle}
\def\b{\mathbf}
\def\R{\mathbf{R}}
\def\Z{\mathbf{Z}}
\def\st{~:~}
\def\bar{\overline}
\def\inv{^{-1}}
\def\deg{^\circ}


%\pointname{pts}
\pointsinmargin
\marginpointname{pts}
\addpoints
\pagestyle{head}
%\printanswers

\hideinstnotes  %comment this line out to show instructor notes.

\firstpageheader{MATH 102}{\bf How to Read}{Due Thursday, September 9}
\runningheader{MATH 102}{}{Fall 2021}


\begin{document}
\noindent {\large\bf Name:} \underline{\hspace{2.5in}}
\vskip 1em
Reading a math textbook can be frustrating and time-consuming. That doesn't mean you shouldn't read though.  The goal of this assignment is for you to try some active reading strategies and then briefly reflect on whether they were helpful. 

Here is what you will do: for any one of your other math classes, find a time to read a section of the textbook prior to the day you will be covering that material in class.  Then:
\begin{itemize}
  \item Before you start reading, ask yourself what the section is going to be about.  What do you think you are going to learn by reading this?
  \item As you are reading, look out for terms and notation you don't understand.  What do you do about that?
  \item After reading each part/section/example, ask yourself whether what you just read makes sense.  What will you do if it doesn't?
  \item You will probably have an example included in the reading.  How are you processing that example?  Are you working it out on paper while reading?  Did you stop reading and think about the example before seeing how the book worked out the problem?  Do you understand why this example was included?
  \item After you have finished reading, think back over what you read.  If a classmate asked you what the section was about, do you think you could tell them?  
  \item Perhaps most importantly: what questions do you have about what you read?  Suppose your instructor asks at the start of class whether there are any questions about the reading.  You want to impress your instructor, so you want to ask a good question.  What are you going to ask?  
\end{itemize}



\vskip 1ex

After you have read the section, I would like you to take 10 minutes to write a very short reflection essay (on the rest of this page and the back) about your experience.  You might answer some of the questions above about what you did, or some of the following prompts: What do you think works well?  Are there things you do while reading that you find especially helpful (that we can share with other students)?  Do you think you will have a better classroom experience after having read the book before class?  Do you think you will be able to study for exams better now that you are familiar with the textbook?

\vskip 1ex

\noindent\textbf{Essay:}




\end{document}
