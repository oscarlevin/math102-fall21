\documentclass[11pt,letterpaper]{article}


\usepackage{fullpage, graphicx, url}
\usepackage{enumerate}
\usepackage{multicol}


\pagestyle{empty}
\thispagestyle{empty}


\begin{document}

\begin{center}{\bf \Large Success in the Mathematical Sciences}

{\bf MATH 102 Fall 2021 (1 credit)}
\end{center}
\vskip 2ex

\noindent\textbf{Professor:} Oscar Levin, Ph.D,~~ Ross 2240D,~~ 351-2380,~~ \url{oscar.levin@unco.edu}

\noindent\textbf{Student Hours:} Mon/Wed 1-2pm; Tu/Th 11am-12pm.  Other times by appointment.

\noindent\textbf{Class Meeting Times}: Thu 9:30 - 10:45 in person in Ross 0274 (basement).

\noindent\textbf{Online Resources:} Through Canvas.  \url{canvas.unco.edu}.

\noindent\textbf{Textbook:} None.

\vskip 2 ex

Welcome to what promises to be an exciting and fun filled semester of MATH 102!  I know you are all eager to get started, but please take a few moments to glance at this syllabus.
\vskip 2ex

\noindent
\textbf{Course Description:} Study topics relevant to the first semester freshman math major and their transition into the academic community at UNC.  Emphasis on learning groups, technology, library utilization, math major exploration, understanding curriculum and graduation requirements, developing a 4-year educational plan, and developing critical thinking and problem-solving skills.  Non-repeatable.

\vskip 2ex
\noindent
\textbf{Course Objectives:} Overall, the purpose of this course is to support you as you transition to college life and UNC so that you can successfully achieve your education and career goals.  This is done, in part, by helping students to:
\begin{enumerate}
  \item Feel a connection to peers, faculty, and the University through involvement on campus.
  \item Make a commitment to your mathematics major.
  \item Navigate the university processes and procedures that are required for graduation.
  \item Understand the expectations UNC has of students as members of an academic community.
  \item Develop skills related to becoming a successful mathematician, such as problem solving, mathematical writing, and the use of technology.
  \item Explore post-graduation options for careers or graduate study related to mathematics.
\end{enumerate}


\vskip 2ex


\noindent
\textbf{Method of Evaluation:} Your success depends on being prepared for and participating in each class.  You are expected to attend every class session on time and to have your homework and reading completed when assigned.  

Your grade for the course will be based on participation and assigned homework and projects.  Assignments will give you an opportunity to learn about topics such as developing a study plan, career options in mathematics, technology tools, and mathematical writing.  All assignments will be posted and submitted through Canvas.

This course is graded on an S/U (satisfactory/unsatisfactory) basis. 

\vskip 1ex

\begin{multicols}{2}
\noindent\textbf{Course Requirements:}

\begin{tabular}{ll}
Participation: & 30\%  \\
Assignments: & 70\% \\
\end{tabular}
\vskip 1em

\noindent\textbf{Grade Scale:}

\begin{tabular}{ll}
75-100\% & S\\
Below 74\% & U \\
\end{tabular}
\end{multicols}
\vskip 2ex
\noindent\textbf{Attendance Policy:} You are expected to attend every class period.  It is impossible to receive participation points if you are not in class.  In case you must miss a class due to illness or other reasonable excuse, we will broadcast the course on Zoom so you can participate remotely.  This is only intended for unusual situations; attending class in person is important to reach the learning objectives of the class.

\vskip 2ex


\noindent{\bf Classroom Policies:} Don't be rude.  Turn off your cell phones when in class and keep them put away, arrive on time, and do not pack up your things before the end of class.  When working in groups, try your hardest to keep the conversation on the mathematics at hand.  

It is also important that you are not rude to each other.  In doing mathematics, or almost anything worth doing in life, one is going to make many errors and false starts while becoming more proficient. Think, for example, of learning to play a musical instrument, or learning an athletic skill, or developing a friendship. We want to establish a classroom atmosphere where the inevitable false starts and mistakes become an opportunity to learn and to get better – not an opportunity for embarrassment. Thus, please be constructive and polite in questioning your colleagues in class. 

\vskip 2 ex

\noindent{\bf Statement of Academic Integrity:} Don't cheat!  It is expected that members of this class will observe strict policies of academic honesty.  In particular, you are expected to solve homework problems by yourself or together with your group, and not find solutions online.  In general, UNC's policies and recommendations for academic misconduct will be followed. For additional information, please see the Student Code of Conduct at the Dean of Student's website \url{http://www.unco.edu/dos/Conduct/codeofconduct.html}. In the case of academic appeals, university procedures will be followed. For information on academic appeals, see \url{http://www.unco.edu/regrec/Current%20Students/AcademicAppeals.html}.

\vskip 2 ex

\noindent\textbf{Disability Resources}: It is the policy and practice of the University of Northern Colorado to create inclusive learning environments.  If there are aspects of the instruction or design of this course that present barriers to your inclusion or to an accurate assessment of your achievement (e.g. time-limited exams, inaccessible web content, use of videos without captions), please communicate this with your professor and contact Disability Resource Center (DRC) to request accommodations.  Office: (970) 351-2289, Michener Library L-80. Students can learn more about the accommodation process at  \url{https://www.unco.edu/disability-resource-center/}.



\end{document}
